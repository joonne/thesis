\documentclass[main.tex]{thesis.tex}
\begin{document}
	
\chapter{Introduction}
\pagenumbering{arabic}
\setcounter{page}{1}

Recommender systems have been successfully utilized to aid customers in decision making.
In fact, they are constantly present in our everyday life.
Whether a customer is shopping online, watching a movie from Netflix, browsing the Facebook or simply reading the news.
All of these tasks involve a presence of a recommendation engine.
Basically all parts of our daily life include recommendations of some sorts.
However, the most basic type of recommendation is the one from human to human and happens completely without computers.
However, humans can only recommend effectively those items they have personally experienced.
This is where recommendation systems (RSs) become useful as they can potentially offer recommendations from thousands of different items.

Recommendation can be divided into two major categories: item-based recommendation and user-based recommendation.
In item-based recommendation the idea is to search similar items, since the user could prefer same kind of items also in the future.
In user-based recommendation the user is thought to be interested in items purchased by similar users thus trying to finds similar users to be able to offer items bought by them.

Apache Spark is a framework for building distributed programs.
A distributed program denotes that the execution of the program is divided between a number of processing nodes.
Recommendation problem can be modeled as an distributed program in which two matrices, users and items, are processed with an iterative algorithm that can be run in parallel.

Spark is built with Scala, a general-purpose, multi paradigm programming language that provides support for functional programming and a strong static type system.
We will be using Scala in the implementation so a short introduction on the programming language will also be provided.

This thesis is structured as follows.
Chapter 2 describes recommendation systems.
Chapter 3 discusses Apache Spark, an open source framework for building distributed programs.
Chapter 4 presents the implementation.
In Chapter 5, we go through the results.
Finally, in Chapter 6 the evaluation is presented along with conclusions.

\end{document}