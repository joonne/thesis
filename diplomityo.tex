%\documentclass[12pt,a4paper,english
% ,twoside,openright
%]{tutthesis}
\documentclass[12pt,a4paper,english]{tutthesis}

% Note that you must choose either Finnish or English here and there in this
% file.
% Other options for document class
  % ,twoside,openright   % If printing on both sides (>80 pages)
  % ,twocolumn           % Can be used in lab reports, not in theses

% Ensure the correct Pdf size (not needed in all environments)
\special{papersize=210mm,297mm}
\setlength\overfullrule{5pt}
\usepackage{amsmath}
\usepackage[all]{xy}
\usepackage{multirow}
\usepackage{subfiles}
\usepackage{caption}



% LaTeX file for BSC/MSc theses and lab reports.
% Requires the class file (=template) tutthesis.cls and figure files,
% either tut-logo, exampleFig (as pdf or eps) and example_code.c
% Author: Sami Paavilainen (2006)
% Modified: Heikki Huttunen (heikki.huttunen@tut.fi) 31.7.2012.
%           Erno Salminen, @tut.fi, 2014-08-15
%             - added text snippets from the writing guide
%             - added lots of comments: both tips and alternative styles
%             - added an example table
%             - and so on...

% More information about Latex basics:
% [Tobias Oetiker, Hubert Partl, Irene Hyna, Elisabeth Schlegl, The
% Not So Short Introduction to LATEX2e, Version 5.03, April 2014, 171
% pages.  Availbale: http://tobi.oetiker.ch/lshort/lshort.pdf]


%
% Define your basic information
%
\author{Altti Hiironen}
\title{�lykk��t sopimukset ja Ethereum-sovellusalusta} % primary title (for front page)
\titleB{Otsikko}     % translated title for abstract
\thesistype{Diplomity�} % or Bachelor of Science, Laboratory Report... 
\examiner{Kuka tarkistaa} % without title Prof., Dr., MSc or such

% Put your thesis' main language last
% http://mirrors.ctan.org/macros/latex/required/babel/base/babel.pdf
\usepackage[english, finnish]{babel}
\usepackage{lmodern}


%
% You can include special packages or define new commands here at the
% beginning. Options are given in brackets and package name is in
% braces:  \usepackage{opt]{pkg_name}

% Option1) for bibliography does not need additional packages.

% Option2b) for bibliography: old way for using Name-year citations
% http://www.ctan.org/tex-archive/macros/latex/contrib/harvard/ 
%\usepackage{harvard}  


% Option3) for bibliography: newer way, esp. for Name-year citations
% http://www.ctan.org/pkg/biblatex
%\usepackage[style=authoryear,maxcitenames=2,backend=bibtex,
%  firstinits=true]{biblatex}
%% Note that option style=numeric works as well
%\addbibresource{thesis_refs.bib}




% You can also add your own commands
\newcommand\todo[1]{{\color{red}!!!TODO: #1}} % Remark text in braces appears in red
\newcommand{\angs}{\textsl{\AA}}              % , e.g. slanted symbol for �ngst�m

% Preparatory content ends here



\pagenumbering{roman} % was: {Roman}
\pagestyle{headings}
\begin{document}



% Special trick so that internal macros (denoted with @ in their name)
% can be used outside the cls file (e.g. \@author)
\makeatletter



%
% Create the title page.
% First the logo. Check its language.
\thispagestyle{empty}
\vspace*{-.5cm}\noindent
\includegraphics[width=8cm]{tty_tut_logo}   % Bilingual logo



% Then lay out the author, title and type to the center of page.
\vspace{6.8cm}
\maketitle
\vspace{7.7cm} % -> 6.7cm if thesis title needs two lines

% Last some additional info to the bottom-right corner
%\begin{flushright}  
%  \begin{minipage}[c]{6.8cm}
%    \begin{spacing}{1.0}
%      \textsf{Tarkastaja: \@examiner}\\
%      %\textsf{xxxxxxx tiedekuntaneuvoston}\\
%      %\textsf{kokouksessa dd.mm.yyyy}\\
%      %\textsf{Examiner: Prof. \@examiner}\\
%      %\textsf{Examiner and topic approved by the}\\ 
%      %\textsf{Faculty Council of the Faculty of}\\
%      %\textsf{xxxx}\\
%      %\textsf{on 30th July 2014}\\
%    \end{spacing}
%  \end{minipage}
%\end{flushright}

% Leave the backside of title page empty in twoside mode
\if@twoside
\clearpage
\fi



%
% Use Roman numbering I,II,III... for the first pages (abstract, TOC,
% termlist etc)
\pagenumbering{Roman} 
\setcounter{page}{0} % Start numbering from zero because command 'chapter*' does page break


% Foreign students do not need Fininsh abstract (tiivistelm�). Move
% this before English abstract if thesis is in Finnish. Move also the
% otherlanguage command to the English abstract (if needed).


\chapter*{Tiivistelm�} % Asterisk * turns numbering off

\begin{spacing}{1.0}
         {\bf \textsf{\MakeUppercase{\@author}}}: \@title\\  % or use \@title when thesis is in Finnish
         \textsf{Tampereen teknillinen yliopisto}\\
         \textsf{ty�, \pageref*{endOfDoc} sivua}\\ %
\end{spacing}

Tiivistelm�





% Add the table of contents, optioanlly also the lists of figures,
% tables and codes.

\renewcommand\contentsname{Sis�llys} % Set Finnish name, remove this if using English
\setcounter{tocdepth}{3}              % How many header level are included
\tableofcontents                      % Create TOC

%\renewcommand\listfigurename{Kuvaluettelo}  % Set Finnish name, remove this if using English
%\listoffigures                               % Optional: create the list of figures
%\markboth{}{}                                % no headers


%\renewcommand\listtablename{Taulukkoluettelo} % Set Finnish name, remove this if using English
%\listoftables                                  % Optional: create the list of tables
%\markboth{}{}                                  % no headers




%\renewcommand\lstlistlistingname{Ohjelmaluettelo} % SetFinnish name, remove this if using English
%\lstlistoflistings                                % Optional: create the list of program codes
%\markboth{}{}                                     % no headers




%
% Term and symbol exaplanations use a special list type
%

\chapter*{Termit}
\markboth{}{}                                % no headers


% You don't have to align these with whitespaces, but it makes the
% .tex file more readable

\begin{table}[h]
	\begin{tabular}{@{}lp{0.58\textwidth}}
		Termi1          &  selitys1 \\ \\
		Termi2         &  Selitys2\\ \\                                       
	\end{tabular}
\end{table}


% The actual text begins here and page numbering changes to 1,2...
% Leave the backside of title empty in twoside mode
\if@twoside
%\newpage
\cleardoublepage
\fi


\renewcommand{\chaptername}{} % This disables the prefix 'Chapter' or
                              % 'Luku' in page headers (in 'twoside'
                              % mode)


\subfile{1-johdanto.tex}

% \subfile{2-lohkoketjuteknologia.tex}

% \subfile{3-ethereum.tex}

% \subfile{4-sovellus.tex}

% \subfile{5-yhteenveto.tex}



%
% The bibliography, i.e the list of references (3 options available)
%
\newpage


% Extra for Finnish theses

\renewcommand{\bibname}{Bibliography}     % Bilingual babel puts Finnish ``Kirjallisuttaa'' otherwise. Strange...
\renewcommand{\bibname}{L�hteet}         % Set Finnish header, remove this if using English
%\addcontentsline{toc}{chapter}{L�hteet}  % Include this in TOC
\addcontentsline{toc}{chapter}{\bibname}  % Include this in TOC


%
% Option1: Write the bibliographic information into .bib file
% (e.g. thesis_refs.bib) and use bibtex tool to do the formatting.
%

% You must execute: pdflatex d_tyo.tex; bibtex d_tyo; pdflatex.tex
% First command creates the cross-refeerence file .aux for bibtex and
% last combines the bibtex output to the rest.  Many styles are
% available, see e.g. at
% http://www.ctan.org/tex-archive/biblio/bibtex/base
% http://www.reed.edu/cis/help/latex/bibtexstyles.html

% 1a) Numeric style:
\bibliographystyle{IEEEtranS}   % the IEEE's sorted numeric style
% List is sorted first by author if present. If not, then by editor,
% organization, title, and last by key.
% http://mirrors.ctan.org/macros/latex/contrib/IEEEtran/bibtex/IEEEtran_bst_HOWTO.pdf

% 1b) Author-year style:
% see http://www.ctan.org/tex-archive/macros/latex/contrib/harvard/

\bibliography{thesis_refs}    % Insert {author,title,year...} info of your reference
\markboth{\bibname}{\bibname} % Set page header

\label{endOfDoc}
%\appendix
%\pagestyle{headings}
% \renewcommand{\appendixname}{Liite} % Extra. Set Finnish prefix for page header
%
%
%
%%
% a) Not-so-handy way, but at least it works
%% 





\end{document}