\documentclass[12pt,a4paper,english]{tutthesis}

\usepackage{mathtools}
\usepackage{listings}
\usepackage{caption}
\usepackage{graphicx}
\graphicspath{ {images/} }
\usepackage{lmodern}
\usepackage{enumitem}
\usepackage{subfiles}
\usepackage{xcolor}

% Ensure the correct Pdf size (not needed in all environments)
\special{papersize=210mm,297mm}

% Define your basic information
\author{Jonne Petteri Pihlanen}
\title{Suosittelijaj�rjestelm�n rakentaminen Apache Sparkilla}      % primary title (for front page)
\titleB{Building a Recommendation Engine with Apache Spark} % translated title for abstract
\thesistype{Diplomity�} % or Bachelor of Science, Laboratory Report... 
\examiner{????} % without title Prof., Dr., MSc or such

% Special trick to use internal macros outside the cls file
% (e.g. \@author). Trick is reversed with \makeatother a bit later.
\makeatletter

% Define the pdf document properties.  Fill in your own keywords.
\hypersetup{   
  pdftitle={\@title},
  pdfauthor={\@author},
  pdfkeywords={recommmendation}
}

\usepackage[finnish]{babel}

\pagenumbering{Roman}
\pagestyle{headings}
\begin{document}

% Create the title page.
\thispagestyle{empty}
\vspace*{-1cm}\noindent
\includegraphics[width=8cm]{tty_tut_logo}

% Then lay out the author, title and type to the center of page.
\vspace{6.8cm}
\maketitle
\vspace{7cm} % modify if thesis title needs many lines

% Last some additional info to the bottom-right corner
\begin{flushright}  
  \begin{minipage}[c]{6.8cm}
    \begin{spacing}{1.0}
      \textsf{Examiner: \@examiner}\\
      \textsf{Examiner and topic approved by the}\\ 
      \textsf{Faculty Council of the Faculty of}\\
      \textsf{xxxx}\\
      \textsf{on 1st September 2014}\\
    \end{spacing}
  \end{minipage}
\end{flushright}

% Leave the backside of title page empty in twoside mode
\if@twoside
\clearpage
\fi

\pagenumbering{roman} 
\setcounter{page}{0} % Start numbering from zero because command 'chapter*' does page break


% Some fields in abstract are automated, namely those with \@ (author,
% title in the main language, thesis type, examiner).

% !!! Problems with other language. Disable it (ES, 2014-11-03)
%%\begin{otherlanguage}{english} %  Following text in in 2nd language
\chapter*{Abstract}

\begin{spacing}{1.0}
         {\bf \textsf{\MakeUppercase{\@author}}}: \@titleB\\   % use \@titleB when thesis is in Finnish
         \textsf{Tampere University of Technology}\\
         \textsf{\@thesistype, xx pages} \\
         \textsf{September 2016}\\
         \textsf{Master's Degree Program in Signal Processing}\\
         \textsf{Major: Data Engineering}\\
         \textsf{Examiner: \@examiner}\\
         \textsf{Keywords: }\\
\end{spacing}

The amount of recommendation engines around the Internet is constantly growing.

This paper studies the usage of Apache Spark when building a recommendation engine.

\chapter*{Tiivistelm�}

\begin{spacing}{1.0}
         {\bf \textsf{\MakeUppercase{\@author}}}: \@titleB\\  % or use \@title when thesis is in Finnish
         \textsf{Tampereen teknillinen yliopisto}\\
         \textsf{Diplomity�, xx sivua}\\ %
         \textsf{syyskuu 2016}\\
         \textsf{Signaalink�sittelyn koulutusohjelma}\\
         \textsf{P��aine: Data Engineering}\\
         \textsf{Tarkastajat: \@examiner}\\
         \textsf{Avainsanat: }\\
\end{spacing}

\makeatother % Make the @ a special symbol again, as \@author and \@title are not neded after this


\chapter*{Preface}

Thanks to my wife, Noora, for pushing me forward with the thesis when my own interest was completely gone. Without you this would never have been ready.

~ 
% Tilde ~ makes an non-breakable spce in LaTeX. Here it is used to get
% two consecutive paragraph breaks

Tampere, 

~


Jonne Pihlanen




%
% Add the table of contents, optionally also the lists of figures,
% tables and codes.
%

\renewcommand\contentsname{Table of Contents} % Set English name (otherwise bilingual babel might break this), 2014-09-01
\renewcommand\contentsname{Sis�llys}         % Set Finnish name
\setcounter{tocdepth}{3}                      % How many header level are included
\tableofcontents                              % Create TOC

%\renewcommand\lstlistlistingname{List of Programs}      % Set English name (otherwise bilingual babel might break this)
%%\renewcommand\lstlistlistingname{Ohjelmaluettelo} % SetFinnish name, remove this if using English
%\lstlistoflistings                                % Optional: create the list of program codes
%\markboth{}{}                                     % no headers


%
% Term and symbol exaplanations use a special list type
%

\chapter*{List of abbreviations and symbols}
\markboth{}{}                                % no headers

% You don't have to align these with whitespaces, but it makes the
% .tex file more readable
\begin{termlist}
	\item [Spark] 					Fast and general engine for large-scale data processing
	\item [Information retrieval (IR)]	Activity of obtaining relevant information resources from a collection of information resources.
\end{termlist}

% Leave the backside of abbreviation list empty in twoside mode
\cleardoublepage

% The actual text begins here and page numbering changes to 1,2...
\newpage             % needed for page numbering
\pagenumbering{arabic}
\setcounter{page}{1} % Start numbering from zero because command
                     % 'chapter*' does page break
\renewcommand{\chaptername}{} % This disables the prefix 'Chapter' or
                              % 'Luku' in page headers (in 'twoside'
                              % mode)

\subfile{1-johdanto.tex}

\subfile{2-suosittelijajarjestelmat.tex}

%\subfile{3-apache-spark.tex}

% \subfile{4-implementation.tex}

%\subfile{5-result.tex}

%\subfile{6-evaluation.tex}

% The bibliography, i.e the list of references (3 options available)
\newpage

\renewcommand{\bibname}{Bibliography}
\addcontentsline{toc}{chapter}{\bibname}
\bibliographystyle{IEEEtranS}
\bibliography{thesis_refs}
\markboth{\bibname}{\bibname}

\end{document}

